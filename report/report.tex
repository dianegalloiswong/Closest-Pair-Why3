\documentclass[11pt, openany, titlepage]{article}

\usepackage{amsmath}
\usepackage{amsfonts}
\usepackage{amssymb}

\usepackage[utf8]{inputenc}
%\usepackage[OT1]{fontenc}
%\usepackage[latin1]{inputenc}
%\usepackage[frenchb]{babel}



\setcounter{section}{1}
\setcounter{secnumdepth}{1}

\begin{document}


{\center
\LARGE\textbf{Divide-and-Conquer Algorithms\\
Application to the Closest Pair Problem}
}

\vspace{3em}

\section{Specification and Brute-Force Algorithm}

\subsection{Questions 1, 2, 3}

Simple réécriture de l'énoncé.


\subsection{Question 4}

\textbf{brute\_force\_search\_sub\_array}
 -- 
%\noindent
La précondition décrit que ${a[low..high-1]}$ est un sous-tableau de $a$ de longueur au moins 2, et \emph{closest\_pair\_post\_for} fournit exactement la postcondition voulue. Les invariants décrivent que $(!min,!f,!s)$ est toujours un triplet de réponse autorisé (indices distincts dans les bornes, $!min$ distance entre les points correspondants), et de plus $!min$ est inférieur aux distances pour les couples de points déjà parcourus dans la boucle considérée. Pour la boucle sur $i$, il s'agit des couples d'indices valides $x$ et $y$ avec au moins un d'eux (sans perdre de généralité $x$) strictement inférieur à $i$. Pour celle sur $j$, les couples d'indices $i$ et $y$, $i+1\leq y < j$. Si on s'en tient là (sans le dernier assert avec $m\_loop\_i$) on n'arrive pas à montrer la conservation de l'invariant de la boucle sur $i$ exprimant que $!min$ est inférieur aux distances déjà étudiées, car pour passer de $i$ à $i+1$ on n'arrive pas à réutiliser le résultat pour $i$ qui porte sur $!min$ qui a pu être modifié au cours de la boucle sur $j$. Une solution consiste à reparler des couples avec $x<i$ dans la boucle sur $j$ (invariant commenté, qui peut remplacer les deux derniers invariants) car pour une étape de cette boucle, le prouveur se rend compte que la nouvelle valeur de $min$ est inférieure ou égale à la précédente. Mais j'avais envie de garder mon invariant ne portant que sur les "nouveaux" couples explorés dans la boucle sur $j$ et de comprendre comment rajouter juste ce qu'il manquait, et cela a fonctionné en ajoutant que $!min$ au cours de la boucle sur $j$ est toujours inférieur à la valeur de $min$ à l'entrée de la boucle.

\

\noindent
\textbf{brute\_force\_search} -- Le tableau doit être de longueur au moins 2, et $closest\_pair\_post$ fournit exactement la postcondition voulue.



\section{Closest Pair in 1D}

\subsection{Question 5}

Un invariant exprimant que $(!min,!f,!f+1)$ est une réponse valide pour le sous-tableau ${a[0..i]}$ a suffi.

\subsection{Question 6}

\end{document}